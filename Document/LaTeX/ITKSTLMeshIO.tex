%
% Complete documentation on the extended LaTeX markup used for Insight
% documentation is available in ``Documenting Insight'', which is part
% of the standard documentation for Insight.  It may be found online
% at:
%
%     http://www.itk.org/

\documentclass{InsightArticle}

\usepackage[dvips]{graphicx}

%%%%%%%%%%%%%%%%%%%%%%%%%%%%%%%%%%%%%%%%%%%%%%%%%%%%%%%%%%%%%%%%%%
%
%  hyperref should be the last package to be loaded.
%
%%%%%%%%%%%%%%%%%%%%%%%%%%%%%%%%%%%%%%%%%%%%%%%%%%%%%%%%%%%%%%%%%%
\usepackage[dvips,
bookmarks,
bookmarksopen,
backref,
colorlinks,linkcolor={blue},citecolor={blue},urlcolor={blue},
]{hyperref}


\title{MeshIO class to read and write STL files in ITK}

%
% NOTE: This is the last number of the "handle" URL that
% The Insight Journal assigns to your paper as part of the
% submission process. Please replace the number "1338" with
% the actual handle number that you get assigned.
%
\newcommand{\IJhandlerIDnumber}{1338}

% Increment the release number whenever significant changes are made.
% The author and/or editor can define 'significant' however they like.
\release{1.00}

% At minimum, give your name and an email address.  You can include a
% snail-mail address if you like.
\author{Luis Ibanez$^{1}$}
\authoraddress{$^{1}$Kitware Inc. Clifton Park, NY}

\begin{document}

%
% Add hyperlink to the web location and license of the paper.
% The argument of this command is the handler identifier given
% by the Insight Journal to this paper.
%
\IJhandlefooter{\IJhandlerIDnumber}


\ifpdf
\else
   %
   % Commands for including Graphics when using latex
   %
   \DeclareGraphicsExtensions{.eps,.jpg,.gif,.tiff,.bmp,.png}
   \DeclareGraphicsRule{.jpg}{eps}{.jpg.bb}{`convert #1 eps:-}
   \DeclareGraphicsRule{.gif}{eps}{.gif.bb}{`convert #1 eps:-}
   \DeclareGraphicsRule{.tiff}{eps}{.tiff.bb}{`convert #1 eps:-}
   \DeclareGraphicsRule{.bmp}{eps}{.bmp.bb}{`convert #1 eps:-}
   \DeclareGraphicsRule{.png}{eps}{.png.bb}{`convert #1 eps:-}
\fi


\maketitle


\ifhtml
\chapter*{Front Matter\label{front}}
\fi


% The abstract should be a paragraph or two long, and describe the
% scope of the document.
\begin{abstract}
\noindent
This document describes the implementation of an ITK class to support the
reading and writing of Meshes in STL format. The Meshes are assumed to contain
2D manifolds embedded in a 3D space. In practice, it would be desirable to use
this class to read and write QuadEdgeMeshes.

This paper is accompanied with the source code, input data, parameters and
output data that the authors used for validating the algorithm described in
this paper. This adheres to the fundamental principle that scientific
publications must facilitate reproducibility of the reported results.

\end{abstract}

\IJhandlenote{\IJhandlerIDnumber}

\tableofcontents

\section{Introduction}

The STL file format is a very common standards for the transmission of Mesh
data.  Originally designed for Stereo Lithography, the STL format has become a
widely used standard for storing and sharing mesh data.

\url{http://en.wikipedia.org/wiki/STL_(file_format)}

Recently, this format has been adopted as the standard way of preparing shapes
as input to 3D printing. Given that ITK is a natural choice for segmenting
anatomical structures from 3D medical data, it is desirable to be able to
generate meshes from such structures, and then store them in STL files suitable
for 3D printing. The class described in this article will make that possible,
using a pure ITK pipeline.


\section{Software Requirements}

You need to have the following software installed:

% The {itemize} environment uses a bullet for each \item.  If you want the 
% \item's numbered, use the {enumerate} environment instead.
\begin{itemize}
  \item  Insight Toolkit 4.5
  \item  CMake 2.8
\end{itemize}

Note that other versions of the Insight Toolkit are also available in the
testing framework of the Insight Journal. Please refere to the following page
for details

\url{http://www.insightsoftwareconsortium.org/wiki/index.php/IJ-Testing-Environment}


%%%%%%%%%%%%%%%%%%%%%%%%%%%%%%%%%%%%%%%%%
%
%  Insert the bibliography using BibTeX
%
%%%%%%%%%%%%%%%%%%%%%%%%%%%%%%%%%%%%%%%%%

\bibliographystyle{plain}
\bibliography{InsightJournal}

\end{document}
